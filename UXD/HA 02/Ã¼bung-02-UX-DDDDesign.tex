\documentclass[a4paper,10pt]{article}

% Hier die Nummer des Blatts und Autoren angeben.
\newcommand{\blatt}{1}
\newcommand{\autor}{ UX-DDDDesign / Maurice Da Cruz Guerreiro, Lara-Marie Ehlen, Henry Sheng-Long Hon, Julian Mühlbauer, Nils Berg }

\usepackage{hci}
\begin{document}
% Seitenkopf mit Informationen
    \kopf

    \title{Webbrowser 2.0}
    \section{Was ist das Web für mich? Wie verwende ich es?}
    \paragraph{Was ist das Web?}
    Das Web ist nicht wegzudenken, ich verwende es für Informationsabruf im Alltag komplett in jedem Aspekt eingebunden.
    Es ist ein Verbindungsstück zwischen mir und meiner Welt, es ist für mich ein Portal um da Leben effizienter zu machen um Information von überall abrufen zu können.
    \paragraph{Wofür wird es konkret angewendet?}
    Bei allen wird es für den Abruf von Informationen für das Studium verwendet.
    Außerdem für Social media, Recherche und Shopping.



    \section{Aufgabe 2: Aktuelle Hausaufgabe UXD einsehen}
    \subsection{Maurice, Protokoll: Nils}
    \begin{enumerate}
        \item Computer starten
        \item Einloggen
        \item Spotlight öffnen
        \item \glqq Sa\grqq{} eintippen und Enter drücken um Safari zu öffnen
        \item Auf das Bookmark \glqq MIN Moodle\grqq{} klicken
        \item Login
        \item UXD Kurs auswählen
        \item Scrollen bis zu aktuellem Übungstermin
        \item Klick auf Übungstermin
        \item Anzeige der Aufgabe
    \end{enumerate}

    \subsection{Henry, Protokoll: Julian}
    \begin{enumerate}
        \item Laptop öffnen
        \item Passwort eingeben
        \item Browser-Icon klicken (Google Chrome)
        \item Bookmark-Liste: Youtube-Icon klicken
        \item Suchleiste: NBA warriors vs magic eintippen, enter
        \item Erstes Video normalerweise Highlights, klick
        \item Skip Ad
        \item F für Fullscreen
    \end{enumerate}

    \section{Aufgabe 3: Aktuelle Hausaufgabe UXD einsehen}
    \subsection{Maurice und Nils}
    \begin{enumerate}
        \item Computer starten
        \item Einloggen
        \item Spotlight mit CMD + Space öffnen
        \item \glqq S\grqq{} als Suchbegriff eingeben
        \item Enter drücken
        \item Safari öffnet sich
        \item Auf MIN Moodle in Favorites klicken
        \item UHH Login Portal öffnen
        \item Login via Apple Password Manager und Touch ID ausfüllen
        \item Login ausführen
        \item Passkey mittels Apple Password Manager und Touch ID freigeben
        \item Startseite MIN Moodle
        \item In kusrübersicht auf UXD Kurs navigieren
        \item Scrollen bis zur aktuellen Woche
        \item Unter \glqq Hausaufgabe\grqq{} auf aktuelle Aufgabe klicken
        \item Aufgabe anzeigen
    \end{enumerate}

    \subsection{Julian, Protokoll: Henry}
    \begin{enumerate}
        \item Öffnen Laptop
        \item Passwort eingeben, enter
        \item Browser Icon unten in der Taskbar klicken
        \item Internetverbindung checken (WLAN auswählen)
        \item Neuer Tab
        \item Dann Youtube-Icon
        \item Search Bar: warriors vs magic, Enter
        \item Etwas Scrollen zum Vergleichen der Videos
        \item Passendes Video anklicken
        \item Skip Ad
        \item Doppelklick für Fullscreen
        \item Video läuft
        \item Lautstärke anpassen
    \end{enumerate}

    \section{Aufgabe 4: Schalftfläche im Browser}
    \subsection{Maurice und Nils}
    \begin{tabular}{|p{5cm}|p{9cm}|}
        \hline
        \textbf{Schaltfläche} & \textbf{Erklärung} \\
        \hline
        Bookmark Bar & Zugriff auf aktiv hinterlegte Bookmarks \\
        \hline
        Sidebar toggle & Blendet Sidebar ein oder aus \\
        \hline
        Tabgruppen Manager Dropdown Menü & Zugriff auf Tabgruppen und neue Tabs, Tabgruppen und Fenster \\
        \hline
        Vor und Zurück Buttons & Geht zur vorherigen oder nächsten Seite \\
        \hline
        Extensions & Zugriff auf genutzte Browser Erweiterungen \\
        \hline
        Searchbar & Adressleiste zum Aufrufen und Suchen von Webseiten, Refresh Button, Zoom, Website Einstellungen, Auf Seite suchen, neues Bookmark anlegen, Seite übersetzen, störende Elemente ausblenden \\
        \hline
        History Button & Anzeige von Verlauf \\
        \hline
        Share Button & Teilen der aktuellen Seite \\
        \hline
        New tab Tab Button & Öffnet neue Tab \\
        \hline
        Show Tab Overview Button & Zeigt Tabübersicht an \\
        \hline
        Tableiste & Zeigt aktuell geöffnete Tabs an \\
        \hline
    \end{tabular}

    \subsection{Julian, Henry und Lara}
    \begin{tabular}{|p{5cm}|p{9cm}|}
        \hline
        \textbf{Schaltfläche} & \textbf{Erklärung} \\
        \hline
        Pfeil nach unten oben links & Tabs durchsuchen \\
        \hline
        Backward, Forward, Refresh & Navigation zu vorheriger/nächster Seite, Seite neu laden \\
        \hline
        Search Bar (Browser) & Suche und Adresseingabe im Browser \\
        \hline
        Settings in Searchbar & Einstellungen in der Suchleiste \\
        \hline
        Stern & Bookmark setzen \\
        \hline
        Install Youtube button & Youtube als App installieren \\
        \hline
        NordVPN & VPN Extension \\
        \hline
        Sonstige extensions & Weitere Browser-Erweiterungen (inkl. Adblock) \\
        \hline
        Downloads & Download-Übersicht \\
        \hline
        Profile (Browser) & Browser-Profil verwalten \\
        \hline
        Settings in Chrome & Chrome-Einstellungen \\
        \hline
        Burger Menu & Hauptmenü \\
        \hline
        Youtube Logo & Zurück zur Youtube-Startseite \\
        \hline
        Search Bar (Youtube) & Suchleiste auf Youtube \\
        \hline
        Search Button & Suchbutton \\
        \hline
        Microphone & Sprachsuche \\
        \hline
        Create Button & Video/Content erstellen \\
        \hline
        Notifications & Benachrichtigungen \\
        \hline
        Profile (Youtube) & Youtube-Profil verwalten \\
        \hline
        Sidebar & Navigation: Home/Shorts/Subscriptions \\
        \hline
        You & Persönliche Youtube-Inhalte \\
        \hline
        Subscriptions & Abonnierte Kanäle \\
        \hline
    \end{tabular}

    \section{Aufgabe 5: Tatsächlich genutzte Funktionen}
    \subsection{Maurice und Nils}
    \paragraph{Maurice zum Abruf der aktuellen UXD Hausaufgabe}
    Maurice hat ausschließlich einmal in der Bookmark Bar auf den enstprechenden Eintrag geklickt.
    Hierfür gibt es einen Punkt, sonst 0.

    \paragraph{Was könnte man bei einem Web Browser 2.0 besser machen?}
    Die vielen kleinen Leisten und Menüs in der Top Bar nehmen vergleichsweise viel vertikalen Platz ein.
    Um hier in der Höhe noch ca. 20 Pixel zur Anzeige der Webseite zu gewinnen, könnten die dauerhaft angezeigte Bookmark Bar und die Tab Bar mit einem ähnlichen Slide-in und Slide-out Effekt wie das macOS Dock versehen werden.
    Durch diese Änderung wäre mehr Platz zur Anzeige gewonnen, nicht so häufig genutzte Funktionen wären immer noch einfach abrufbar und das Verhalten wäre inenrhalb des Betriebssystems konsistent.
    Alternativ könnten die Top bars, ähnlich zur Verwendung auf Mobilgeräten beim Hochscrollen ausgeblendet oder minimiert werden um Platz zu sparen.

    \subsection{Julian, Henry und Lara}
    \paragraph{Henry zum }
    Es wird je einmal die Seach Bar von Browser und Youtube benutzt.
    Sonst wird gar keine Funktion genutzt.
    Für jede der Search Bars gibt es 1 Punkt, sonst 0 Punkte.

    \paragraph{Was könnte man bei einem Web Browser 2.0 besser machen?}
    Möglicherweise wäre die Kombination beider Search Bars um direkt auf die Seite mit den Suchergebnissen zu kommen.
    Man könnte in der ersten Suchseite die YouTube Suche als Vorschlag anzeigen.

\end{document}
