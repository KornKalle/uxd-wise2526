\documentclass[a4paper,10pt]{article}

% Hier die Nummer des Blatts und Autoren angeben.
\newcommand{\blatt}{1}
\newcommand{\autor}{ UX-DDDDesign / Maurice Da Cruz Guerreiro, Lara-Marie Ehlen, Henry Sheng-Long Hon, Julian Mühlbauer, Nils Berg }

\usepackage{hci}
\begin{document}
% Seitenkopf mit Informationen
    \kopf

    \title{Webbrowser 2.0}
    \section{Was ist das Web für mich? Wie verwende ich es?}
    \paragraph{Was ist das Web?}
    Das Web ist nicht wegzudenken, ich verwende es für Informationsabruf im Alltag komplett in jedem Aspekt eingebunden.
    Es ist ein Verbindungsstück zwischen mir und meiner Welt, es ist für mich ein Portal um da Leben effizienter zu machen um Information von überall abrufen zu können.
    \paragraph{Wofür wird es konkret angewendet?}
    Bei allen wird es für den Abruf von Informationen für das Studium verwendet.
    Außerdem für Social media, Recherche und Shopping.



    \section{Aufgabe 2: Aktuelle Hausaufgabe UXD einsehen}
    \subsection{Maurice, Protokoll: Nils}
    \begin{enumerate}
        \item Computer starten
        \item Einloggen
        \item Spotlight öffnen
        \item \glqq Sa\grqq{} eintippen und Enter drücken um Safari zu öffnen
        \item Auf das Bookmark \glqq MIN Moodle\grqq{} klicken
        \item Login
        \item UXD Kurs auswählen
        \item Scrollen bis zu aktuellem Übungstermin
        \item Klick auf Übungstermin
        \item Anzeige der Aufgabe
    \end{enumerate}

    \subsection{Henry, Protokoll: Julian}
    \begin{enumerate}
        \item
    \end{enumerate}

    \section{Aufgabe 3: Aktuelle Hausaufgabe UXD einsehen}
    \subsection{Maurice, Protokoll: Nils}
    \begin{enumerate}
        \item Computer starten
        \item Einloggen
        \item Spotlight mit CMD + Space öffnen
        \item \glqq S\grqq{} als Suchbegriff eingeben
        \item Enter drücken
        \item Safari öffnet sich
        \item Auf MIN Moodle in Favorites klicken
        \item UHH Login Portal öffnen
        \item Login via Apple Password Manager und Touch ID ausfüllen
        \item Login ausführen
        \item Passkey mittels Apple Password Manager und Touch ID freigeben
        \item Startseite MIN Moodle
        \item In kusrübersicht auf UXD Kurs navigieren
        \item Scrollen bis zur aktuellen Woche
        \item Unter \glqq Hausaufgabe\grqq{} auf aktuelle Aufgabe klicken
        \item Aufgabe anzeigen
    \end{enumerate}

    \subsection{Henry, Protokoll: Julian}
    \begin{enumerate}
        \item
    \end{enumerate}

    \section{Aufgabe 4: Schalftfläche im Browser}
    \begin{tabular}{|l|l|}
        \hline
        \textbf{Schaltfläche} & \textbf{Erklärung} \\
        \hline
        Bookmark Bar & Zugriff auf aktiv hinterlegte Bookmarks \\
        \hline
        Sidebar toggle & Blendet Sidebar ein oder aus \\
        \hline
        Tabgruppen Manager Dropdown Menü & Zugriff auf Tabgruppen und neue Tabs, Tabgruppen und Fenster \\
        \hline
        Vor und Zurück Buttons & Geht zur vorherigen oder nächsten Seite \\
        \hline
        Extensions & Zugriff auf genutzte Browser Erweiterungen \\
        \hline
        Searchbard & Ausblenden von störendenen Elementen, Zoom, Website Einstellungen, Auf Seite suchen, neues Bookmark für aktuelle Seite anlegen, Adressleiste zum Aufrufen und Suchen von Webseiten, Seite übersetzen, Refresh Button \\
        \hline
        History Button & Anzeige von Verlauf \\
        \hline
        Share Button & Teilen der aktuellen Seite \\
        \hline
        New tab Tab Button & Öffnet neue Tab \\
        \hline
        Show Tab Overview Button & Zeigt Tabübersicht an \\
        \hline
        Tableiste & Zeigt aktuell geöffnete Tabs an \\
    \end{tabular}

    \section{Aufgabe 5: Tatsächlich genutzte Funktionen}
    \paragraph{Maurice zum Abruf der aktuellen UXD Hausaufgabe}
    Maurice hat ausschließlich einmal in der Bookmark Bar auf den enstprechenden Eintrag geklickt.
    Hierfür gibt es einen Punkt, sonst 0.

    \subparagraph{Was könnte man bei einem Web Browser 2.0 besser machen?}
    Die vielen kleinen Leisten und Menüs in der Top Bar nehmen vergleichsweise viel vertikalen Platz ein.
    Um hier in der Höhe noch ca. 20 Pixel zur Anzeige der Webseite zu gewinnen, könnten die dauerhaft angezeigte Bookmark Bar und die Tab Bar mit einem ähnlichen Slide-in und Slide-out Effekt wie das macOS Dock versehen werden.
    Durch diese Änderung wäre mehr Platz zur Anzeige gewonnen, nicht so häufig genutzte Funktionen wären immer noch einfach abrufbar und das Verhalten wäre inenrhalb des Betriebssystems konsistent.

\end{document}
