\documentclass[a4paper,10pt]{article}

% Hier die Nummer des Blatts und Autoren angeben.
\newcommand{\autor}{ Nils Berg }

\usepackage{hci}
\begin{document}
% Seitenkopf mit Informationen
    \kopf
    \title{
        Confidence, not consistency, characterizes flashbulb memories
    }

    \section{Fragestellung der Studie}
    Das Paper mit dem Titel "Confidence, not consistency, characterizes flashbulb memories", wurde im Jahr 2003 in dem Magazin Psychological Science veröffentlicht.
    In diesem gehen Jennifer Talarico und Davic Rubin der Frage nach ob Erinnerungen an emotionale oder überraschende Erinnerungen genauer und konsister sind als andere Erinnerungen.
    Sie nutzen dafür den Begriff "Flashbulb Memory".
    Die Forschenden stellen dabei konkret 3 Fragen, welche sie in ihrer Studie überprüft haben.
    \begin{itemize}
        \item Sind "Flashbulb Memories" konsistener als alltägliche Erinnerungen?
        \item Gibt es einen Zusammenhang zwischen stärkerem Wiedererleben ("increased recollection") und dem Glauben an ihre Korrektheit?
        \item Welche Rolle spielen Emotion und deren Intensität beim Erleben des Ereignisses dabei?
    \end{itemize}

    \section{Versuchsaufbau}
    Die Studie startete am 12. September 2001 mit 54 Studierenden der Duke University.
    Die Studierenden wurden nach ihrer Erinnerung bezüglich der Terroranschläge vom 11. September 2001 am vorherigen morgen befragt.
    Die Teilnehmenden wurden anschließend randomisiert in 3 Gruppen unterteilt, welche zu je einem Folgetermin in logarithmisch gleichbleibenden Abständen eingeladen wurden.
    Diese Termine waren 1 Woche, 6 Wochen und 32 Wochen nach dem 11. September 2001.
    \\
    Bei jedem dieser Termine wurden den Teilnehmenden in der ersten Hälfte spezifische Fragen gestellt, wie sie die Ereignisse des 11. September 2001 erlebt haben.
    In der zweiten Hälfte wurden Fragen nach einem Alltagsevent am gleichen Tag gestellt.
    Beim 2. Termin wurde zusätzlich ein PTSD-Symptomfragebogen beantwortet.

    \section{Ergebnisse}
    Die Autor:innen kommen zu dem Ergebnis, dass kein objektiver Unterschied beim Verblassen zwischen "Flashbulb Memories" und "Everyday Memories" besteht.
    Es gab in ihrer Studie keinen signifikanten Unterschied bei der Anzahl konsistenter und inkonsistenter Details.
    \\
    Gleichwohl wurden die Flashbulb Memories subjektiv als lebhafter, emotionaler und sicherer empfunden.
    Dies führte dazu, dass bei den Teilnehmenden der Glaube an die Korrektheit bei diesen Erinnerungen gegenüber den Alltagserinnerung stabil blieb, während sie sich objektiv veränderte.
    \\
    Flashbulb Memories zeichneten sich durch eine höhere emotionale Intensität aus, diese sagte jedoch nichts über die Konsistenz der Erinnerungen,
    sondern lediglich etwas über den Glauben an die Genauigkeit aus.
    \\
    Stärkere viszerale Reaktionen lassen sich als Vektor für spätere PTSD Symptome deuten.

    \section{Einordnung}
    Durch die Erkenntnisse, dass Flashbulb Memories subjektiv eine nachhaltig höhere Sicherheit und Lebhafigkeit vermitteln,
    objektiv betrachtet aber genauso unbeständig wie Alltagserinnerungen sind, lassen sich ältere Studienergebnisse
    (bspw. von Brown \& Kulik \textit{(1977). “Flashbulb Memories.”}\textit{Cognition, 5, 73–99.}) widerlegen,
    welche Flashbulb Memories als besonders genaue Erinnerungen interpretieren.
    \\
    Emotion wirkt den Studienergebnisse nach nicht als Indikator für Genauigkeit der Erinnerungen, sondern als Indikator für Vertrauen in deren Korrektheit.
    \\
    Durch diese Erkenntnisse werden rekonstruktive Gedächtnistheorien
    (bspw. \textit{Bartlett, F. C. (1932).} Remembering: A Study in Experimental and Social Psychology. Cambridge: Cambridge University Press.)
    gestützt, wonach Erinnerungen narrative sind,
    die sich verändern aber subjektiv kohärent sind.
    \\
    Mit dieser Erkenntnis lässt sich beispielsweise erklären, warum Menschen beim Erinnerung oft Vertrauen mit Genaugkeit verwechseln.
    Auch für den Umgang mit Zeug:innenaussagen lassen sich hieraus Erkenntnisse ableiten.


\end{document}
