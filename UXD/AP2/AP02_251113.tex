\documentclass[a4paper,10pt]{article}

% Hier die Nummer des Blatts und Autoren angeben.
\newcommand{\autor}{ Nils Berg }

\usepackage{hci}
\begin{document}
% Seitenkopf mit Informationen
    \kopf
    \title{
        Semantic Integration of Verbal Information into a Visual Memory
    }

    \section{Fragestellung der Studie}
    Das Paper mit dem Titel \textit{Semantic Integration of Verbal Information into a Visual Memory} wurde im Jahr 1978 im Journal of Experimental Psychology veröffentlicht.
    In diesem gehen Elizabeth Loftus, David Miller und Helen Burns der Frage nach, wie nachträgliche verbale Informationen die visuelle Erinnerung an ein Ereignis verändern.
    Die Forschenden stellen dabei drei Leitfragen, welche sie in ihrer Studie überprüft haben:
    \begin{itemize}
        \item Werden nachträglich ergänzte Informationen in die ursprüngliche Erinnerung integriert?
        \item Führen irreführende Informationen zu geringerer Erkennungsgenauigkeit?
        \item Wann ist der Einfluss irreführender Informationen am stärksten?
    \end{itemize}

    \section{Versuchsaufbau}
    Insgesamt nahmen 1.242 Proband:innen an fünf Experimenten, sowie einer Pilotstudie teil.
    In allen Experimenten wurde den Teilnehmenden eine Abfolge von Farbdias gezeigt, die den Hergang eines Unfalls zwischen einem Auto und einem Fußgänger darstellten.
    Das kritische Dia zeigte entweder ein \textit{Stoppschild} oder ein \textit{Vorfahrt gewähren}-Schild.
    Anschließend erhielten die Proband:innen Fragen, die konsistente, irreführende oder neutrale zusätzliche Informationen enthielten.
    Abschließend wurde jeweils ein visueller Erkennungstest durchgeführt, um den Einfluss der verbalen Information auf die Erinnerung zu prüfen.

    \subsection{Pilotstudie}
    In der Pilotstudie wurden 30 Folien der Unfallszene gezeigt, wobei das Auto entweder an einem Stoppschild oder an einem Vorfahrt gewähren-Schild hielt.
    Direkt im Anschluss beantworteten die Teilnehmenden einen Fragebogen, dessen zentrale Frage die Existenz eines dieser Schilder voraussetzte und somit konsistente oder irreführende Information bot.
    Nach einer 20-minütigen Filler-Aufgabe folgte ein Ja/Nein-Erkennungstest, bei dem beide kritischen Dias vorkamen.
    Die Studie zeigte, dass irreführende Information die Trefferrate senkten.
    Da augrund der Ähnlichkeit der beiden Dias Positionseffekte auftraten, wurden in den Hauptstudien Forced-Choice Tests durchgeführt wurden.

    \subsection{Experiment 1}
    195 Teilnehmende sahen die 30 Folien und beantworteten anschließend einen Fragebogen mit 20 Fragen.
    Frage 17 enthielt entweder konsistente oder irreführende Information zum Schild.
    Nach einer 20-minütigen Pause folgte ein Forced-Choice-Test mit 15 Bildpaaren, von denen eines das kritische Paar (Stoppschild vs. Vorfahrt gewähren-Schild) war.
    Die Proband:innen mussten das zuvor gesehene Bild auswählen.

    \subsection{Ergebnisse}
    Nach hinzufügen von konsistenten Informationen erkannten 75\% das richtige Bild, nach irreführenden Informationen nur 41\%.
    Die Ergebnisse zeigten, dass nachträgliche Informationen in die ursprüngliche Gedächtnisrepräsentation integriert werden.

    \subsection{Experiment 2}
    90 Personen erhielten entweder konsistente, irreführende oder neutrale zusätzliche Informationen nach Ansicht des Unfallhergangs.
    Nach einer 20-minütigen Pause erfolgte derselbe Forced-Choice-Test wie in Experiment 1.
    Anschließend gaben die Teilnehmenden an, welches Schild sie gesehen und welches der Fragebogen suggeriert hatte.

    \subsection{Ergebnisse}
    Nur 12\% derjenigen, die falsch geantwortet hatten, erkannten die Irreführung, was als Argument für eine tatsächliche Gedächtnisverzerrung gelten soll.

    \subsection{Experiment 3}
    In diesem Experiment sollte untersucht werden, zu welchem Zeitpunkt das Hinzufügen von irreführenden Informationen den größten Einfluss hat.
    720 Personen sahen den Unfallhergang und erhielten den Fragebogen entweder sofort nach der Präsentation oder vor dem Erkennungstest.
    Der Test erfolgte nach fünf verschiedenen Zeitintervallen (0 Minuten, 20 Minuten, 1 Tag, 2 Tage oder 1 Woche) und enthielt zusätzlich die Frage nach der Selbsteinschätzung der Erinnerungsgenauigkeit.

    \subsection{Ergebnisse}
    Irreführende Informationen reduzierten die Rate der korrekten Erinnerung deutlich, insbesondere in der Gruppe, welche die Informationen erst kurz vor dem Test erhielten.
    Dies wurde als Argument dafür, dass mit zeitlich größerem Abstand hinzugefügte irreführende Erinnerungen die ursprünglich Erinnerung besonders stark verzerren und dafür, dass mit größerem zeitlichen Abstand der Glauben an falsche Erinnerungen steigt.

    \subsection{Experiment 5}
    Im letzten Experiment wurde überprüft, ob der Effekt auch für andere gezeigte Objekte gilt.

    80 Personen sahen einen neuen aus 20 Folien bestehenden Foliensatz eines anderen Unfalls mit vier kritischen Objekten, die jeweils in zwei Varianten vorkamen (z.B. Ski oder Schaufel).
    Anschließend lasen sie einen Augenzeugenbericht, der für einige der Objekte irreführende Informationen enthielt.
    Nach einer Pause folgte ein Forced-Choice-Test mit 10 Bildpaaren.

    \subsection{Ergebnisse}
    Die korrekte Erkennung sank bei irreführender Information von 70,8\% auf 55,3\%.
    Dies wurde als Beleg für den Effekt der Integration falscher verbaler Information in visuelle Erinnerung interpretiert.

    \section{Ergebnisse}
    Die Ergebnisse zeigten konsistent über alle Experimente, dass nachträgliche verbale Informationen in die ursprüngliche Erinnerung integriert werden.
    Irreführende Informationen reduzierten die Erkennung und/oder Wiedergabe der korrekten Informationen signifikant.
    Wohingegen das hinzufügen konsistener Informationen die Erkennungsrate erhöhte.
    \\
    Dieser Effekt bestand auach dann noch, wenn die Teilnehmenden von der möglichkeit hinzugefügter Fehlinformationen wussten.
    Am stärksten zeigte sich dieser Effekt, wenn die zusätzlichen Informationen direkt vor dem Test hinzugefügt wurden.
    Ebenso zeigte sich, dass die Teilnehmenden subjektiv eine hohe Sicherheit an die Korrektheit der Erinnerung empfanden, auch wenn dies objektiv falsch war.

    \section{Einordnung}
    Die Studie liefert einen empirischen Beleg dafür, dass Erinnerungen rekonstruktiv  sindund durch das hinzufügen von Informationen veränderbar sind.
    Durch nachträgliche Präsentation von verbalen Informationen, können diese in visuelle Erinngerungen integriert werden und deren Wahrheitsgehalt negativ beeinflussen.
    \\
    Damit möchte die Studie die Idee eines fotografischen Gedächtnisses widerlegen und gleichzeitig die Theorie des rekonstruktiven Erinnerns nach Bartlett von 1932 stützen.
    Außerdm liefert sie einen Beleg für die praktische Relevanz der Aussagenpsychologie, da suggestive Befragungstechniken Erinnerungen real verzerren könnten.

\end{document}
