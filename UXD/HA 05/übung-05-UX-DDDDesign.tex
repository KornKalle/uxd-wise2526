\documentclass[a4paper,10pt]{article}

% Hier die Nummer des Blatts und Autoren angeben.
\newcommand{\blatt}{5}
\newcommand{\autor}{ UX-DDDDesign / Maurice Da Cruz Guerreiro, Lara-Marie Ehlen, Henry Sheng-Long Hon, Julian Mühlbauer, Nils Berg }

\usepackage{hci}
\usepackage{xcolor}
\usepackage{tikz}
\usetikzlibrary{shapes,arrows,positioning}

\begin{document}
% Seitenkopf mit Informationen
    \kopf

    \section*{Schlüsselpfad-Szenario: Posten eines Reels auf Instagram}

    \subsection*{1. Episodisches Szenario}
    
    Marie ist 28 Jahre alt und arbeitet als Lifestyle Influencerin kommend aus Hamburg.
    An einem sonnigen Samstagmorgen besucht sie während eines Berlin Urlaubs den Mauerpark, um das bunte Treiben des Mauerpark Flohmarkts zu erleben.
    Während sie durch die Stände schlendert, entdeckt sie einen Straßenmusiker, der eine tolle Performance darbietet.
    Sie nimmt ihr iPhone und filmt ein etwas 2 minütiges Video der Performance.
    Zurück in ihrer AirBnB Unterkunft öffnet Marie die Instagram-App auf ihrem iPhone und tippt auf das Plus-Symbol am unteren Bildschirmrand.
    Sie wählt die Option „Reel" aus dem Menü aus.
    Marie wählt das zuvor aufgenommene Video aus ihrer Galerie aus und beginnt mit der Bearbeitung.
    Sie schneidet das Video auf die besten 25 Sekunden zu und fügt einen beliebten Song aus der Instagram-Musikbibliothek hinzu.
    Anschließend wendet sie einen Farbfilter an, um die Atmosphäre des Marktes besser einzufangen und die Farben wärmer erscheinen zu lassen..
    Marie fügt noch ein Overlay hinzu: „Sonntagsvibes am Mauerpark" mit einer animierten Schriftart.
    Sie taggt den Standort „Mauerpark Berlin" und fügt relevante Hashtags wie \#StreetMusic und \#BerlinLife hinzu.
    Nach einer kurzen Vorschau ist sie zufrieden mit dem Ergebnis und tippt auf „Teilen".
    Innerhalb von Minuten erhält ihr Reel die ersten Likes und Kommentare von ihren Follower:innen.

    \subsection*{2. Konzeptionelles Szenario}
    
    Das System ermöglicht es Usern, kurze Videoformate zu erstellen und zu teilen.
    Der User initiiert den Prozess durch Auswahl der Reel-Erstellungsfunktion in der Hauptnavigation.
    Das System bietet zwei Hauptoptionen: Direktaufnahme über die App-Kamera oder Import bestehender Videos aus der Gerätebibliothek.
    Nach der Videoauswahl präsentiert das System eine umfassende Bearbeitungsumgebung.
    Diese umfasst Werkzeuge zur zeitlichen Anpassung des Videomaterials durch Trimmen und Schneiden.
    Weitere Werkzeuge sind entweder voreingestellte Farbfilter oder möglichkeiten zur detaillierteren Anpassung der Farben.
    Das System stellt eine kuratierte Musikbibliothek zur Verfügung, aus der Audio-Tracks ausgewählt und synchronisiert werden können.
    Visuelle Anpassungen erfolgen durch Anwendung von Filtern und Effekten aus einer vordefinierten Bibliothek.
    Das System ermöglicht das Hinzufügen von Textebenen mit verschiedenen Schriftarten und Animationsoptionen.
    Nutzer können Sticker, GIFs und andere grafische Elemente aus der integrierten Bibliothek einfügen.
    Das System bietet Funktionen zur Kennzeichnung von Standorten und anderen Nutzern mittels Tags und Verlinkungen des Usernames.
    Vor der Veröffentlichung können Hashtags zur Kategorisierung und Reichweitensteigerung hinzugefügt werden.
    Das System ermöglicht eine Vorschau des finalen Reels in Echtzeit.
    Nach Bestätigung wird der Content hochgeladen und im Nutzerprofil sowie im Feed der Follower veröffentlicht.
    Das System verarbeitet das Video, optimiert es für verschiedene Endgeräte und macht es innerhalb des Netzwerks innerhalb kurzer Zeit auffindbar.

    \subsection*{3. Epic}
    
    Als \textcolor{blue}{Content Creator:in} möchte ich \textcolor{red}{Kurzvideos erstellen und mit meiner Community teilen}, um \textcolor{green}{meine Reichweite zu erhöhen und mit meinem Publikum zu interagieren}.

    \subsection*{4. User Story}
    
    Als \textcolor{blue}{Instagram-User} möchte ich \textcolor{red}{ein bearbeitetes Video als Reel posten}, um \textcolor{green}{meine Erlebnisse mit meinen Followern zu teilen}.

    \subsection*{5. Hierarchical Task Analysis (HTA)}
    
    \begin{center}
    \begin{tikzpicture}[
        node distance=0.4cm and 0.5cm,
        maintask/.style={rectangle, draw, thick, text width=3.2cm, align=center, minimum height=0.55cm, fill=gray!10, font=\footnotesize},
        task/.style={rectangle, draw, text width=2.8cm, align=center, minimum height=0.48cm, font=\tiny},
        subtask/.style={rectangle, draw, text width=2.3cm, align=center, minimum height=0.42cm, font=\tiny},
        minisubtask/.style={rectangle, draw, text width=2cm, align=center, minimum height=0.38cm, font=\tiny},
        plan/.style={font=\tiny\itshape, text width=2.5cm, align=left}
    ]
    
    % Level 0: Hauptziel
    \node[maintask] (0) {0. Reel auf Instagram posten};
    \node[plan, right=of 0, xshift=1cm] (p0) {Plan 0: 1-2-3-4-5};
    
    % Level 1: Task 1
    \node[task, below=of 0, yshift=-0.2cm] (1) {1. Instagram-App starten};
    
    % Level 1: Task 2
    \node[task, below=of 1, yshift=-0.2cm] (2) {2. Reel-Funktion aufrufen};
    \node[subtask, right=of 2, xshift=0.3cm] (21) {2.1 Plus-Symbol tippen};
    \node[subtask, right=of 21] (22) {2.2 „Reel" auswählen};

    % Level 1: Task 3
    \node[task, below=of 2, yshift=-0.2cm] (3) {3. Video vorbereiten};
    \node[subtask, right=of 3, xshift=0.3cm] (31) {3.1 Galerie öffnen};
    \node[subtask, right=of 31] (32) {3.2 Video selektieren};
    
    % Level 1: Task 4
    \node[task, below=of 3, yshift=-0.2cm] (4) {4. Video bearbeiten};
    \node[subtask, right=of 4, xshift=0.3cm] (41) {4.1 Video zuschneiden};
    \node[subtask, right=of 41] (42) {4.2 Audio hinzufügen};
    \node[subtask, right=of 42] (43) {4.3 Visuelle Effekte};
    \node[subtask, right=of 43] (44) {4.4 Text-Overlays};
    
    % Level 3: Subtasks für 4.2
    \node[minisubtask, below=of 42, yshift=-0.75cm] (421) {4.2.1 Musikbiblio. öffnen};
    \node[minisubtask, right=of 421] (422) {4.2.2 Song wählen};
    
    % Level 3: Subtasks für 4.3
    \node[minisubtask, below=of 43, yshift=-0.15cm] (431) {4.3.1 Filter anwenden};
    \node[minisubtask, right=of 431] (432) {4.3.2 Farbkorrektur};
    
    % Level 1: Task 5
    \node[task, below=of 4, yshift=-2.1cm] (5) {5. Reel publizieren};
    \node[subtask, right=of 5, xshift=0.3cm] (51) {5.1 Metadaten eingeben};
    \node[subtask, right=of 51] (52) {5.2 Vorschau prüfen};
    \node[subtask, right=of 52] (53) {5.3 Teilen};
    
    % Level 3: Subtasks für 5.1
    \node[minisubtask, below=of 51, yshift=-0.1cm] (511) {5.1.1 Standort taggen};
    \node[minisubtask, right=of 511] (512) {5.1.2 Hashtags};
    \node[minisubtask, right=of 512] (513) {5.1.3 Beschreibung};
    
    % Connections Level 0-1
    \draw[-] (0) -- (1);
    \draw[-] (1) -- (2);
    \draw[-] (2) -- (21);
    \draw[-] (21) -- (22);
    \draw[-] (2) -- (3);
    \draw[-] (3) -- (31);
    \draw[-] (31) -- (32);
    \draw[-] (3) -- (4);
    \draw[-] (4) -- (41);
    \draw[-] (41) -- (42);
    \draw[-] (42) -- (43);
    \draw[-] (43) -- (44);
    \draw[-] (42) -- (421);
    \draw[-] (421) -- (422);
    \draw[-] (43) -- (431);
    \draw[-] (431) -- (432);
    \draw[-] (4) -- (5);
    \draw[-] (5) -- (51);
    \draw[-] (51) -- (52);
    \draw[-] (52) -- (53);
    \draw[-] (51) -- (511);
    \draw[-] (511) -- (512);
    \draw[-] (512) -- (513);
    
    \end{tikzpicture}
    \end{center}

\end{document}
